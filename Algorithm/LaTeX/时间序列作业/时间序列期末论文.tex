\documentclass[12pt, a4paper]{ctexart}
\usepackage{fancyhdr}
\usepackage{graphicx}
\usepackage{harpoon}
\usepackage{ctex}
\usepackage{amsmath}
\pagestyle{plain}
\usepackage{mathrsfs}
\usepackage{amssymb}

\title{非平稳时间序列的主要方法和研究意义}
\author{10161511403 魏森辉}
\date{}
  
% 注释区
% 包括论文题目 摘要 主要章节 参考文献
% 单栏不少于六页
% $\qquad$缩进 \bigskip换行

\begin{document}
\maketitle{}
\tableofcontents % 产生文档目录
\thispagestyle{empty} %目录页不显示页码
\newpage
\setcounter{page}{1} % 从下面开始编页码

\begin{flushleft}
    
\section{引言}
\subsection{时间序列概述}

在统计学中,一个变量在一定连续时点或一定连续时期上测量的观测值的集合称为时间序列 。
时间序列的基本要素: 是被研究现象所属的时间范围。
是反映该现象在一定时间条件下数量特征的值,即在不同时间上的统计。数据时间可以是年份、季度、月份或其他任何时间形式。
非平稳序列是指包含趋势性、季节性或周期性等特性的序列,它可能只含有其中的一种成分,也可能是几种成分的组合。

在许多领域,人们日益重视对各种现象的定量观测和有关数据的收集和分析。这些数据一般按时间顺序排列,由于受到多种偶然因素的影响,往往表现出某种随机性,且观测值之间存在着相互依赖关系。对这种按时间顺序排列的动态数据进行研究,构成了数理统计的一个重要分支——时间序列分析

时间序列根据所研究的依据不同,有不同的分类
当按时间序列的统计特性来分类时,有平稳时间序列和非平稳时间序列。若一个时间序列的概率分布与时间$t$无关,我们称该时间序列为严格的(狭义的)平稳时间序列。如果时间序列的一、二阶矩存在,而且对任意时刻$t$满足:(1)均值为常数,(2)协方差为时间间隔$\tau$的函数,则称该时间序列为宽平稳时间序列,也叫广义平稳时间序列。反之,我们把不具有平稳性(时间序列的均值或协方差是与时间有关)的时间序列称为非平稳序列

由于在实际问题中,我们遇到的时间序列,特别是反映社会、经济现象的序列,多数情况并不平稳,而是具有明显的增长或减少趋势,或者含有依时间周期变化的趋势,所以本文主要讨论非平稳时间序列


时间序列预测在目标跟踪、天气预报、市场分析和故障诊
断领域中有广泛的应用.传统的预测方法大都采用线性模型
来近似地表达预测对象的发展规律.如最常用的 AR 模型预
测或ARMA 模型预测, 就是在时间序列平稳的假设基础上, 对
其建立线性模型, 然后采用模型外推的方法预测其未来值.因
此, 这些方法只适用于平稳时间序列的预测.然而, 实际应用
中的时间序列往往是高度非平稳的时间序列, 传统的预测方
法无法取得很好的预测效果


\section{主要方法}
\subsection{AR模型预测方法}
$\qquad$时间序列$\{x_t | t = 1, 2, \dots, N\}$的$AR(n)$模型表示为:\[
x_{t}=\varphi_{1} x_{t}-_{1}+\varphi_{2} x_{t-2}+\dots+\varphi_{n} x_{t-n}+  a_{t}
\]式中,$a_t$称为模型参数,n称为模型的阶数.

$\qquad$对给定的$\{x_t\}$建立$AR(n)$模型后,必须进行适用性检验,其中最根本的检验$a_t$是否为白噪声.可对$\{a_t\}$作一下两方面的检查:

$\qquad$(1)检验$a_t$是否与$a_{t - 1},a_{t -2}, \dots$无关;

$\qquad$(2)检验$a_t$是否与$x_{t - 1},x_{t -2}, \dots$无关;

$\qquad$对$AR(n)$模型进行参数估计和适用性检验后,就可以用所建立的$AR(n)$模型对时间序列$\{x_t\}$进行预测

$\qquad$定义$\hat{x_t}(l)$为在$t$时刻对未来$l$步的预测值,$e_t(l)$为预测误差,即:
\[
e_{t}(l)=x_{t+l}-\hat{x_{t}}(l) \tag{1}
\]
并称预测误差$e_t (l)$的方差为最小时的$\hat{x_t}(l)$值为最佳预测.对于式(1)所定义的$AR(n)$模型,其最佳预测值的计算式如下:

   \[ 
    \left\{
    \begin{array}{ll}
    {0} & {u=0} \\ 
    \sum_{i=1}^{l - 1} \varphi_{\hat{x}_{t}}(l-i)+\sum_{i=1}^{n} \varphi_{i} \hat{x}_{t}+l-i & {1 < l \leqslant n}\\
    \sum_{i=1}^{l - 1} \varphi_{\hat{x}_{t}}(l-i)+\sum_{i=1}^{n} \varphi_{i} \hat{x}_{t}+l-i & {l \ge n}
    \end{array}\right.
\]




\subsection{小波分析}
通过小波分解可以将某些非平稳时间序列分解成多
层近似意义上的平稳时间序列, 然后采用自回归模型对分解后的时间序列进行预测, 从而得到原始时间序列的预测
值.对年平均太阳黑子数的预测结果表明, 该方法比传统的时间序列预测方法和神经网络预测方法的预测精度高, 可
以很好地应用于某些非平稳时间序列的预测中.

\section{研究意义}
\subsection{应用}


序列平稳性是时间序列变形分析建模的重要前提,可以通过对时间序列模型及其特性进行分析,探讨非平稳序列的平稳化问题,对解决实际问题有很大的意义。例如,在隧道变形分析中,变形监测不仅能监视工程构筑物的安全状态,而且对反馈设计和施工质量等起到重要指导作用,其变形分析结果也是对设计数据的验证,为改进设计和科学研究提供资料。变形分析的方法繁多,时间序列分析是一种动态变形分析方法,它从统计自相关的角度来研究随机数据序列的规律。序列平稳性是时间序列建模的重要前提,如果序列非平稳,就要采用合适的方法进行序列的平稳化转换
\subsection{我的}
\qquad 答:大师傅






\end{flushleft}

\end{document}

