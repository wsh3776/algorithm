\documentclass[12pt, a4paper]{ctexart}
\usepackage{fancyhdr}
\usepackage{graphicx}
\usepackage{harpoon}
\usepackage{ctex}
\usepackage{amsmath}
\pagestyle{plain}
\usepackage{mathrsfs}
\usepackage{amssymb}
\usepackage{amsthm}

% 页面边距设置
\usepackage[a4paper,left=2.5cm,right=2.5cm,top=2.5cm,bottom=2.5cm]{geometry}

% 设置行间距的大小
\usepackage{setspace} 
\renewcommand{\baselinestretch}{1.5} 


\newtheorem{define}{定义} %在导言区使用,定义环境名

\title{数学分析辅导讲义}
\author{魏森辉}
\date{2019/12/13}
  
% y ^ {\prime} 莱布尼茨公式


\begin{document}
%\maketitle{}

\section{证明题}

    \begin{flushleft} % 设置左对齐
    

    例: $ f(x) = (x-1) ^ 2$,求 $ f^{\prime} (2x)$ 

    要清楚是对谁求导

    \[
    \begin{aligned}  
        f^{\prime} (x)= &(f(x)) ^ {\prime}  \\
        f^{\prime} (2x) \neq& (f(2x))^{\prime}  \\
        f^{\prime}(2x)=  &  \frac{ (f(2x)) ^ {\prime} }{ 2 } = \frac{ ((2x-1)^2) ^ {\prime} }{ 2 } = 4x - 2
    \end{aligned}
    \]

    莱布尼茨公式\quad 设$u(x),v(x)$有$n$阶导数,则
    \[
(u \cdot v)^{(n)}=\sum_{k=0}^{n} \mathrm{C}_{n}^{k} u^{(k)} v^{(n-k)}
\]
    

    5.4 8.设函数$y = f(x) \text{在点}x \text{处三阶可导,且}f^{\prime}(x) \neq 0, \text{若} f(x) \text{存在反函数} x = f^{-1}(y), \text{试用}f^{\prime} (x), f^{\prime \prime}(x) \text{以及} f^{\prime \prime \prime}(x) \text{表示}(f^{-1})^{\prime \prime \prime}(y).$

    解:要清楚是关于谁求导 
   \[
    \begin{aligned} \frac{\mathrm{d} x}{\mathrm{d} y} &=\frac{1}{\frac{\mathrm{d} y}{\mathrm{d} x}}=\frac{1}{f^{\prime}(x)} \\ 
    \frac{\mathrm{d}^{2} x}{\mathrm{d} y^{2}} &=\frac{\mathrm{d}}{\mathrm{d} y}\left(\frac{\mathrm{d} x}{\mathrm{d} y}\right)=
    \frac{\mathrm{d}}{\mathrm{d} x}\left(\frac{\mathrm{d} x}{\mathrm{d} y}\right) \frac{\mathrm{d} x}{\mathrm{d} y}=\frac{-f^{\prime \prime}(x)}{\left(f^{\prime}(x)\right)^{2}} \cdot \frac{1}{f^{\prime}(x)} =-\frac{f^{\prime \prime}(x)}{\left(f^{\prime}(x)\right)^{3}} \\ 
    \frac{\mathrm{d}^{3} x}{\mathrm{d} y^{3}} &=\frac{\mathrm{d}}{\mathrm{d} y}\left(\frac{\mathrm{d}^{2} x}{\mathrm{d} y^{2}}\right)=\frac{\mathrm{d}}{\mathrm{d} x}\left(\frac{\mathrm{d}^{2} x}{\mathrm{d} y^{2}}\right) \frac{\mathrm{d} x}{\mathrm{d} y}=-\frac{f^{\prime \prime \prime}(x) f^{\prime}(x)-3\left(f^{\prime \prime}(x)\right)^{2}}{\left(f^{\prime}(x)\right)^{4}} \cdot \frac{1}{f^{\prime}(x)} \\ &=\frac{3\left(f^{\prime \prime}(x)\right)^{2}-f^{\prime}(x) f^{\prime \prime \prime}(x)}{\left(f^{\prime}(x)\right)^{5}} \\ 
     \end{aligned}
    \]

    即 
    $$
    (f^{-1})^{\prime \prime \prime}(y)=\frac{3\left(f^{\prime \prime}(x)\right)^{2}-f^{\prime}(x) f^{\prime \prime \prime}(x)}{\left(f^{\prime}(x)\right)^{5}}
    $$

    \bigskip

    
    5.4 9.设$y = \arctan x.$

    (1) 证明它满足方程$(1+x^2)y^{\prime \prime} + 2xy^{\prime} = 0;$

    (2) $y^{(n)} | _ {x = 0}.$
    
    解:

    (1) $y ^ {\prime} = \frac{ 1 }{ 1 + x ^2 },$得$(1 + x^2) y ^ {\prime} = 1$,两边求导,得到
    \[
\left(1+x^{2}\right) y^{\prime \prime}+2 x y^{\prime}=0
\]


    (2) 对两边求$n - 2$次导数,得到
    \[
\begin{aligned}\left(1+x^{2}\right) y^{(n)}+(n-2) 2 x y^{(n-1)}+\frac{(n-2)(n-3)}{2} & 2 y^{(n-2)} \\+2 x y^{(n-1)}+(n-2) 2 y^{(n-2)} &=0 \quad(n \geqslant 3) \end{aligned}
\]

    令$x = 0$得到
    $y^{(n)} | _{x= 0} = -(n - 1)(n - 2)y^{(n - 2)} | _ {x= 0} \quad (n \geqslant 3)$

    所以
    \[
\left.y^{(n)}\right|_{x=0}=\left\{\begin{array}{ll}{(-1)^{\frac{n-1}{2}}(n-1) !y^{\prime}(0),} & {n \text{为奇数}} \\ {(-1)^{\frac{n-2}{2}}(n-1) ! y^{\prime \prime}(0),} & {n \text{为偶数}}\end{array}\right.
\]

    又$y^{\prime}(0) = 1, y^{\prime \prime}(0) = 0,$ 所以
    \[
\left.y^{(2 m)}\right|_{x=0}=0,\left.y^{(2 m+1)}\right|_{x=0}=(-1)^{m}(2 m) !
\]

    5.4 10.设$y = \arcsin x$

    (1)证明它满足方程
    \[
\left(1-x^{2}\right) y^{(n+2)}-(2 n+1) x y^{(n+1)}-n^{2} y^{(n)}=0
\]

    (2)求$y^{(n)} | _ {x = 0}.$

    解:与上一题类似,先对y求一次导,得到关系式$\sqrt{1 - x ^ 2} y ^ {\prime} = 1 $
    在两边求导得到    
    \[
\begin{array}{c}{\sqrt{1-x^{2}} y^{\prime \prime}-\frac{x}{\sqrt{1-x^{2}}} y^{\prime}=0} \end{array}
\]
化简得
\[
\begin{aligned}  
    {\left(1-x^{2}\right) y^{\prime \prime}-x y^{\prime}=0} 
\end{aligned}
\]
后面过程与上题类似,不再赘述

    5.4 11. 证明函数
    \[
f(x)=\left\{\begin{array}{ll}{e^{-\frac{1}{x^{2}}},} & {x \neq 0} \\ {0,} & {x=0}\end{array}\right.
\]
在 $ x = 0$处$n$阶可导且$f^{(n)}(0) = 0,$其中$n$为任意正整数.

    证:当$x \neq 0$ 时,$f^{\prime}(x) = \frac{ 2 }{ x ^ 3 } e ^{-\frac{ 1 }{ x^2 }}$,而
    \[
\begin{aligned} f^{\prime}(0) &=\lim _{x \rightarrow 0} \frac{f(x)-f(0)}{x-0} \\ &=\lim _{x \rightarrow 0} \frac{e^{-\frac{1}{x^{2}}}}{x} \\ &=\lim _{t \rightarrow \infty} \frac{t}{e^{t^{2}}} \quad\left(t=\frac{1}{x}\right) \\ &=0 \end{aligned}
\]

所以 \[
f^{\prime}(x)=\left\{\begin{array}{ll}{\frac{2}{x^{3}} e^{-\frac{1}{x^{2}}},} & {x \neq 0} \\ {0,} & {x=0}\end{array}\right.
\]

同理可得
\[
f^{\prime \prime}(x)=\left\{\begin{array}{ll}{\left(\frac{-6}{x^{4}}+\frac{4}{x^{6}}\right) e^{-\frac{1}{x^{2}}},} & {x \neq 0} \\ {0,} & {x=0}\end{array}\right.
\]

数学归纳法

假设
\[
f^{(n)}(x)=\left\{\begin{array}{ll}{P_{n}\left(\frac{1}{x}\right) e^{-\frac{1}{2}},} & {x \neq 0} \\ {0,} & {x=0}\end{array}\right.
\]
其中$P_n (\frac{ 1 }{ x })$ 为$ \frac{ 1 }{ x }$的$3n$次多项式,则当$x \neq 0$时,
\[
f^{(n + 1)}(x)=\left(\frac{2}{x^{3}} P_{n}\left(\frac{1}{x}\right)-\frac{1}{x^{2}} P_n ^{\prime}\left(\frac{1}{x}\right)\right) \mathrm{e}^{-\frac{1}{x^{2}}}=P_{n+1}\left(\frac{1}{x}\right) \mathrm{e}^{-\frac{1}{x^{2}}}
\]  
显然$P_{n+1} (\frac{ 1 }{ x })$ 为$ \frac{ 1 }{ x }$的$3(n+1)$次多项式
又
\[
\begin{aligned} f^{(n+1)}(0) &=\lim _{x \rightarrow 0} \frac{f^{(n)}(x)-f^{(n)}(0)}{x-0} \\ &=\lim _{x \rightarrow 0} \frac{P_{n}\left(\frac{1}{x}\right) e^{-\frac{1}{x^{2}}}}{x} \\ &=\lim _{t \rightarrow \infty} \frac{t P_{n}(t)}{e^{t^{2}}} \quad\left(t=\frac{1}{x}\right) \\ &=0 \end{aligned}
\]

所以
\[
f^{(n+1)}=\left\{\begin{array}{ll}{P_{n+1}\left(\frac{1}{x}\right) e^{-\frac{1}{x^{2}}},} & {x \neq 0} \\ {0,} & {x=0}\end{array}\right.
\]

由数学归纳法知,$f(x)$在 $ x = 0$处$n$阶可导且$f^{(n)}(0) = 0,$
    \end{flushleft}



\end{document}

