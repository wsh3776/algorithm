\documentclass[12pt, a4paper]{ctexart}
\usepackage{fancyhdr}
\usepackage{graphicx}
\usepackage{harpoon}
\usepackage{ctex}
\usepackage{amsmath}
\pagestyle{plain}
\usepackage{mathrsfs}
\usepackage{amssymb}
\usepackage{amsthm}
\usepackage[a4paper,left=3cm,right=3cm,top=2cm,bottom=2cm]{geometry} % paperwidth=11cm,scale=0.8
\usepackage{setspace} 
\renewcommand{\baselinestretch}{1.5} % 设置行间距的大小


\newtheorem{define}{定义} %在导言区使用,定义环境名

\title{数学分析辅导讲义}
\date{日期 2019/11/15}
\author{魏森辉}
  

\begin{document}
\maketitle{}

\section{计算题}

    \begin{flushleft} % 设置左对齐
    1.求极限$$\lim \limits _{x \rightarrow 0} \frac {\sin{4x}}{\sqrt{x+1}-1}$$

    解:有理化或等价替换($\sqrt[k]{x+1}\sim 1+kx$)\\
    答案:8
    
    2.求极限$$\lim \limits _{\alpha \rightarrow \beta} \frac{e^\alpha - e^\beta}{\alpha - \beta}$$
    
    解:换元$t = \alpha - \beta$
    \\答案:$e^{\beta}$

    3.求极限\[\lim \limits _{n \rightarrow \infty}\frac{{n^2\sin n \sqrt[n]{2}}(1-\cos \frac{1}{n^2})} {\cos \frac{1}{n}(\sqrt{n^2+1}-n)(\frac{1}{n}+1)}\]

    解:取极限时如果结果是非零常数的项可以直接取这个值,如果出现0,则这一项不能轻易取值,要考虑等价无穷小

    答案:原式$=\lim \limits _{n \rightarrow \infty}{ (\sqrt{n^2+1}+n)n^3 \frac{1}{2n^4}}=1$

    4.求极限\[\lim \limits _{x \rightarrow 0} \frac{(3+2 \sin x)^x - 3^x}{tan ^2x}
    \]

    解:不是$1^\infty$型,两个指数形式的相减,往往需要提出公因数,这题类似上面换元的题目,这里需要用到$a^b=e^{blna}$

    答案:

    \[
    \begin{aligned}
    \lim \limits _{x \rightarrow 0} \frac{(3+2 \sin x)^x - 3^x}{tan ^2x} &= \lim \limits _{x \rightarrow 0} {\frac{3^x\left((1+\frac{2}{3}\sin x)^x-1\right)} {\tan^2 x}} \\
    &=\lim \limits _{x \rightarrow 0} \frac{3^x(e^{xln(1+\frac{2}{3}\sin x) }-1)}{\tan^2x} 
    &=\lim \limits _{x \rightarrow 0} \frac{3^xxln(1+\frac{2}{3} \sin x)}{\tan^2 x}
    & = \frac{2}{3}
    \end{aligned}
    \]

    5.多项式求极限模型(看次数最高的项)
    \[
    \lim _{n \rightarrow \infty} \frac{a_{m} n^{m}+a_{m-1} n^{m-1}+\cdots+a_{1} n+a_{0}}{b_{k} n^{k}+b_{k-1} n^{k-1}+\cdots+b_{1} n+b_{0}}=\left\{\begin{array}{ll}{\frac{a_{m}}{b_{m}},} & {k=m} \\ {0,} & {k>m}\end{array}\right.
    \] % \{是特殊字符,\left是定界符

    \[\lim \limits _{n \rightarrow \infty} \frac{3n^3+n}{2n^3+n^2}=\frac{3}{2}\]
    \[\lim \limits _{x \rightarrow +\infty} \frac{\sqrt{x+\sqrt{x+\sqrt{x}}}}{\sqrt{2x+1}}=\frac{\sqrt{2}}{2}\]
    \[\lim \limits _{x \rightarrow \infty} \frac{\sqrt[3]{x^5+x^3+x}+2x^2}{x^{\frac{5}{3}}}=\infty\]
    这个其实不算多项式,但是可以用个放缩,如果是趋于0呢?要先用换元$n=\frac{1}{x}$

    放缩法

    6.\[\lim \limits _{n \rightarrow \infty} \sqrt[n]{1-\frac{1}{n}}\]
    1. 求$
    \lim _{x \rightarrow 0^{+}}\left(1+x+x^{2}\right)^{\sin \frac{1}{x}}
    $的极限

    解:
    \[
    \left(1+x+x^{2}\right)^{-1} \leqslant\left(1+x+x^{2}\right)^{\sin \frac{1}{x}} \leqslant\left(1+x+x^{2}\right)^{1}
    \]

    由迫敛性,得$$\lim _{x \rightarrow 0^{+}} (1+x+x^2)^{\sin \frac{1}{x}} = 1$$
    


    2. 无穷大减无穷大不能判断结果
    \[
    \lim _{x \rightarrow \infty} \frac{x^{2}-5}{x^{2}-1}=1
    \]
    \end{flushleft}

\section{证明题}
    \begin{proof} % 宏包amsthm
    For simplicity, we use
    \[
    E=mc^2
    \]
    That's it.
    \end{proof}

\end{document}

