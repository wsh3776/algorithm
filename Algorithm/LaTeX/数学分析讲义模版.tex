\documentclass[12pt, a4paper]{ctexart}
\usepackage{fancyhdr}
\usepackage{graphicx}
\usepackage{harpoon}
\usepackage{ctex}
\usepackage{amsmath}
\pagestyle{plain}
\usepackage{mathrsfs}
\usepackage{amssymb}
\usepackage{amsthm}
\usepackage[a4paper,left=3cm,right=3cm,top=2cm,bottom=2cm]{geometry} % paperwidth=11cm,scale=0.8

\newtheorem{define}{定义} %在导言区使用,定义环境名

\title{数学分析辅导讲义}
\date{日期 2019/11/15}
\author{XXX}
  

\begin{document}
\maketitle{}

\section{概念题}

    \begin{define}
    若映射$\phi(x)$满足从$X \to X$,则称此映射为变换
    \end{define}


    \begin{flushleft} % 设置左对齐
    1.\[
    \lim _{n \rightarrow \infty} \frac{a_{m} n^{m}+a_{m-1} n^{m-1}+\cdots+a_{1} n+a_{0}}{b_{k} n^{k}+b_{k-1} n^{k-1}+\cdots+b_{1} n+b_{0}}=\left\{\begin{array}{ll}{\frac{a_{m}}{b_{m}},} & {k=m} \\ {0,} & {k>m}\end{array}\right.
    \] % \{是特殊字符,\left是定界符

    \[\lim _{n \rightarrow \infty} \frac{a_{m}n^{m-k}+a_{m-1}n^{m-1-k}+\cdots+a_1n^{1-k}+a_0n^{-k}}{b_kn^k+b_{k-1}n^{k-1}+\cdots+b_1n^{}+b_0}=\left\{\begin{array}{ll}
        {\frac{a_m}{b_{m}}},&{k = m} \\ {0,}& {k>m}
    \end{array}\right.
    \]

    \[
\left\{\begin{array}{ll}{\frac{a_{m}}{b_{m}},} & {k=m} \\ {0,} & {k>m}\end{array}\right.
\]
    1. 求$
    \lim _{x \rightarrow 0^{+}}\left(1+x+x^{2}\right)^{\sin \frac{1}{x}}
    $的极限

    解:
    \[
    \left(1+x+x^{2}\right)^{-1} \leqslant\left(1+x+x^{2}\right)^{\sin \frac{1}{x}} \leqslant\left(1+x+x^{2}\right)^{1}
    \]

    由迫敛性,得$$\lim _{x \rightarrow 0^{+}} (1+x+x^2)^{\sin \frac{1}{x}} = 1$$


    2. 无穷大减无穷大不能判断结果
    \[
    \lim _{x \rightarrow \infty} \frac{x^{2}-5}{x^{2}-1}=1
    \]
    \end{flushleft}

\section{证明题}
    \begin{proof} % 宏包amsthm
    For simplicity, we use
    \[
    E=mc^2
    \]
    That's it.
    \end{proof}

\end{document}

